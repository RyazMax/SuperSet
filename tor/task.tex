
\begin{titlepage}
\singlespacing
\thispagestyle{empty}

\ifx\wauthor\undefined\providecommand{\wauthorspace}{\vspace{7pt}}\fi

\providecommand{\wauthorspace}{}
\providecommand{\wtype}
{{\huge ЗАДАНИЕ}

\bigskip

{\LARGE
на выполнение курсового проекта
}

\bigskip
}
\providecommand{\wauthor}{}
\providecommand{\wgroup}{}
\providecommand{\wfirstlead}{}
\providecommand{\wleadprefix}{Руководитель курсового проекта}



\begin{minipage}{\textwidth}
\fontsize{10}{15}
\centering
\bfseries

Министерство науки и высшего образования Российской
Федерации\\

Федеральное государственное бюджетное образовательное учреждение\\

высшего образования\\

<<Московский государственный технический университет
имени Н.Э. Баумана\\

(национальный исследовательский университет)>>\\

(МГТУ им. Н.Э. Баумана)\\

\end{minipage}

{\vspace{5pt}
\rule{\linewidth}{3pt}
%\Xhline{3pt}

\par
\rule[20pt]{\linewidth}{1pt}
%\Xhline{1pt}
\vspace{-2em}
}

\begin{adjustbox}{minipage=0.4\textwidth,valign=T,right}
        \footnotesize
    \centering
    УТВЕРЖДАЮ

    \begin{adjustbox}{minipage=0.7\textwidth}
        Заведующий кафедрой
    \end{adjustbox}
    \begin{adjustbox}{minipage=0.25\textwidth}
        \uline{\hfillИУ7\hfill}
    \end{adjustbox}\\
    \begin{adjustbox}{minipage=0.25\textwidth, right}
        \scriptsize
        \vspace{-1em}
        (индекс)
    \end{adjustbox}\\
    \begin{adjustbox}{minipage=0.4\textwidth}
        \vspace{.7em}
        \uline{\hfill}
    \end{adjustbox}
    \begin{adjustbox}{minipage=0.55\textwidth}
        \uline{\hfill И.В. Рудаков\hfill}
    \end{adjustbox}\\
    \begin{adjustbox}{minipage=0.45\textwidth, right}
        \scriptsize
        \vspace{-0.6em}
        (И.О.Фамилия)
    \end{adjustbox}\\
    \begin{adjustbox}{minipage=\textwidth, center}
        <<\uline{\hfill}>>\uline{\hfill\hfill\hfill} 20\uline{\hfill} г.
    \end{adjustbox}

\end{adjustbox}

\bigskip

\begin{center}
    \LARGE\textbf{ЗАДАНИЕ}

    \large\textbf{на выполнение курсового проекта}
    \small

    по дисциплине
    \uline{\hfill
        Базы данных
    \hfill}

    Студент группы
    \uline{\hfill
        \wgroup
    \hfill}

    {\singlespacing
    \uline{\hfill
        Рязанов 
        Максим 
        Сергеевич
    \hfill}

    {\scriptsize\vspace{-0.3em}(Фамилия, имя, отчество)}
    }

    \vspace{1em}

    Тема курсового проекта
    \uline{\hfill Сервис для разметки данных \hfill}


    {\flushleft
    Направленность КП (учебный, исследовательский, практический,
    производственный, др.) 
    \hfill}

    \uline{\hfill
        практический
    \hfill}

    Источник тематики (кафедра, предприятие, НИР)
    \uline{\hfill
        кафедра
    \hfill}

    График выполнения проекта: 
        25\% к \uline{\hfill4\hfill} нед.,
        50\% к \uline{\hfill7\hfill} нед.,
        75\% к \uline{\hfill11\hfill} нед.,
        100\% к \uline{\hfill14\hfill} нед.

    \textbf{Задание}
    %\uline{\hfill
     %   Используя язык программирования Golang и базу данных Tarantool разработать веб-приложение, позволяющее загружать и размечать наборы данных разных
     %   видов для последующего использования размеченных данных в задачах машинного обучения.
   % \hfill}
    \uline{На основе базы данных Tarantool с использованием языка программирования Golang разработать веб-приложение, 
    позволяющее загружать и размечать наборы данных разных видов для последующего использования размеченных данных
    в задачах машинного обучения.        
    Обеспечить поддержку двух видов пользователей системы:
    1. администраторов, имеющих возможность конфигурировать тип размечаемых данных и проводимой разметки, настраивать 
    доступ других пользователей к набору данных, принадлежащему администратору.  
    2. обычных пользователей, имеющих возможность производить разметку наборов данных к которым пользователь имеет доступ. 
    Наличие доступа определяется в соответствии с настройками, заданными администратором набора данных.
    Cреди поддерживаемых типов входных данных должны быть: текст, изображения, табличные данные(с задаваемым 
    количеством колонок).
    Среди поддерживаемых типов разметки должны быть: классификация(из заданных пользователем классов), 
    численная метка(целое, вещественное), текст, прямоугольная область(для изображений).
    Предоставить возможность пользователям-администраторам загружать наборы данных для разметки, а также 
    выгружать размеченные данные в форматах csv и json, как через веб-интерфейс, так и с помощью HTTP API для интеграции, 
    разрабатываемого приложения, с системами машинного обучения. 
    }

    {\flushleft
    \textbf{Оформление курсового проекта:}
    \hfill}

    {\flushleft
    Расчетно-пояснительная записка на 
    \uline{\hspace{1em}15-20\hspace{1em}}
    листах формата A4.
    \hfill}

    {\flushleft
        Перечень графического (иллюстративного) материала
        (чертежи, плакаты, слайды и т.п.)
    \hfill}

    \uline{\hfill
        На защиту проекта должна быть предоставлена презентация,
        состоящая из 15-20 слайдов.
    \hfill}

    \uline{\hfill
        На слайдах должны быть отражены: постановка задачи,
        схемы базы данных,
    \hfill}

    \uline{\hfill
        схемы и описания компонент приложения, интерфейс.
    \hfill}

    \uline{\hfill\hfill}

    {\flushleft
        Дата выдачи задания <<\hspace{1.5em}>>
        \uline{\hspace{4em}} 
        20\uline{\hspace{1em}} г.
    \hfill}

\end{center}

\begin{adjustbox}{minipage=0.35\textwidth,valign=T}
    \begin{flushleft}
        \footnotesize
        \textbf{\wleadprefix}
    \end{flushleft}
        \vspace{0em}
    \begin{flushleft}
        \footnotesize
        \textbf{Студент}
    \end{flushleft}
\end{adjustbox}
\hfill
\begin{adjustbox}{minipage=0.3\linewidth,valign=T}
        \vspace{7pt}

    \begin{center}
        \footnotesize
        \uline{\hfill}

        \scriptsize
        (Подпись, дата)
    \end{center}
    \begin{center}
        \footnotesize
        \uline{\hfill}

        \scriptsize
        (Подпись, дата)
    \end{center}
\end{adjustbox}
\hfill
\begin{adjustbox}{minipage=0.3\linewidth,valign=T}
    \wauthorspace\begin{center}
        \footnotesize
        \uline{\hfill\wfirstlead\hfill}

        \scriptsize
        (И.О.Фамилия)
    \end{center}
    \begin{center}
        \footnotesize
        \uline{\hfill\wauthor\hfill}

        \scriptsize
        (И.О.Фамилия)
    \end{center}
\end{adjustbox}
\begin{flushleft}
\uline{Примечание}: Задание оформляется в двух экземплярах: один выдается
студенту, второй хранится на кафедре.
\end{flushleft}

\end{titlepage}
