\section*{Введение}
\addcontentsline{toc}{section}{Введение}

Последнее время всю большую популярность набирает машинное обучение, а также различные прикладные
сервисы на его основе. Одним из этапов обучения модели в машинном обучении является сбор и разметка данных. Данная задача осложняется
тем, что не все типы и наборы данных размечены и находятся в открытом доступе, что иногда требует ручного труда асессоров. Разметка
данных асессорами локально, осложняет агрегацию данных, полученных от разных людей. Данную проблему может решить использование веб-сервиса,
который будет обеспечивать выдачу данных некоторому числу асессоров, а также централизованное хранение полученных меток. 

Целью курсового проекта является создание веб-сервиса для разметки данных,
предоставляющего возможность загружать данные различных типов(изображения, текст, табличные данные), размечать загруженные данные нескольким
пользователям одновременно, а также выгружать метки в форматах JSON, СSV.

Для достижения поставленной цели необходимо выполнить следующие задачи 

\begin{itemize}
    \item Выделить основные сущности предметной области;
    \item на основе выделенных сущностей разработать схему базы данных;
    \item разработать иерархию классов бизнес-логики и доступа к данным;
    \item разработать интерфейс веб-приложения;
    \item реализовать веб-приложение;
    \item обеспечить тестирование реализованного приложения;
\end{itemize}