\section{Технологический раздел}

\subsection{Серверная часть}

Ниже приведены технологии, которые были использованы в серверной части приложения

\begin{itemize}
    \item Go 1.15\cite{godoc} --- язык программирования. 
    Он имеет достаточно много удобных библиотек для веб-разработки и
    обширное сообщество, что уменьшает вероятность возникновения 
    трудноразрешимых проблем, ускоряет разработку и позволяет 
    сконцентрироваться на архитектуре проекта.

    \item Tarantool 2.2.4\cite{tarantooldoc} --- Резидентная СУБД и сервер приложений.
    Приемуществами является скорость работы, 
    огромное количество возможностей,
    хорошая документация и открытый исходный код.
    Недостатков замечено не было.

    \item Go-tarantool\cite{go-tarantool} --- Библиотека для подключения к Tarantool из Go.
    
    \item TarantoolQueue\cite{tarantooldoc} --- Модуль консистентных 
    очередей для Tarantool.

    Использовался для обеспечения консистентной обработки запросов на разметку, предотвращения 
    повторной разметки одного
    элемента выборки.
    
    \item TarantoolDDL\cite{tarantooldoc} --- Модуль декларативного описания таблиц в Tarantool. 
\end{itemize}

\subsection{Клиентская часть}

Инструменты, использованные для браузерной части приложения

\begin{itemize}
    \item HTML5 --- язык разметки веб-страниц.
    
    \item GoTemplates --- язык шаблонов для HTML страниц. Использовался для создания динамических веб-страниц.\cite{gotemplate}

    \item CSS --- язык стилей, предоставляет широкие возможности по 
            манипуляции внешним видом веб-страницы.

    \item JavaScript --- язык программирования, который широко 
        используется для создания интерактивных и динамических 
        веб-страниц.  

    \item Bootstrap\cite{bootstrap4docs} --- Css-фреймфорк, 
        который предоставляет большое количество готовых
        CSS классов, что значительно ускоряет создание
        современного интерфейса.
\end{itemize}

\subsection{Структура проекта}

\begin{itemize}
    \item Models --- модуль, содержащий реализацию сущностей приложения на языке Go.
    \item Auth ---   модуль, авторизации
    \item Repos --- модуль, содержащий логику доступа к данным
        \item Grant --- модуль доступа к данным привелегий
        \item LabeledTask --- модуль доступа к размеченным данным
        \item Project --- модуль доступа к данным проекта
        \item Schema --- модуль доступа к схемам проектов
        \item Session --- модуль доступа к сессиям пользователей
        \item Task --- модуль доступа к размечаемым данным
        \item User --- модуль доступа к данным пользователей
    \item Universe --- модуль - глобальный репозиторий, обеспечивающий централизованную работу с подключениями к базе данных
    \item Tarantool --- модуль обеспечивающий загрузку СУБД Tarantool, создание таблиц, загрузку хранимых процедур на Lua.
\end{itemize}


